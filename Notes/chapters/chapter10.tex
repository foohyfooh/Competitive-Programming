\section{Miscellaneous}

\subsection{Bit Shifting}
\begin{lstlisting}[style=CStyle]
1 << i reperesents 2 raised to that power
\end{lstlisting}


\subsection{Last Set Bit}
\begin{lstlisting}[style=CStyle]
bit = x & -x
\end{lstlisting}

\subsection{Set Bits Counting - Brian Kernighan's Algorithm}
Subtraction of 1 from a number toggles all the bits (from right to left) until the rightmost set bit (including the rightmost set bit). So if we subtract a number by 1 and do bitwise \& with itself (n \& (n-1)), we unset the rightmost set bit. If we do n \& (n-1) in a loop and count the no of times loop executes we get the set bit count.
With this implementation, the number of times it loops is equal to the number of set bits in a given integer.
\begin{lstlisting}[style=CStyle]
int count = 0, n;
while(n != 0){
	n &= (n - 1);
	count++;
}
\end{lstlisting}


\subsection{Binary Counter}
\lstinputlisting[style=CStyle]{./code/binary_counter.cpp}


\subsection{Symmetric Matrix}
\begin{lstlisting}

\end{lstlisting}

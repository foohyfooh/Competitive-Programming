\section{Maths}

\subsection{Greatest Common Divisor (GCD)}
Use the Euclidean Algorithm to determine the gcd of two numbers
\lstinputlisting[style=CStyle]{./code/gcd.cpp}
\lstinputlisting[style=PyStyle]{./code/gcd.py}


\subsection{Extended Euclidean Algorithm}
This builds on the Euclidean Algorithm to find $x$ and $y$ where $d = gcd(a, b) = ax + by$
\lstinputlisting[style=PyStyle]{./code/gcd_extended.py}


\subsection{Modular Exponentiation}
Determining $a^b \ mod \ n$ \\
In the following code, the array b[] is the bits of the power b.
\lstinputlisting[style=CStyle]{./code/modular_exponentiation.cpp}


\subsection{Fastest Exponentiation}
Determining $a^b$
\lstinputlisting[style=CStyle]{./code/fastest_exponentiation.cpp}


\subsection{Matrices}
\lstinputlisting[style=CStyle]{./code/matrix_functions.cpp}


\subsection{Fibonacci}
Fibonacci is defined as \\ \\
$
F_n =
\left\{
\begin{array}{ll}
0                      & \mbox{if } n =    0 \\
1                      & \mbox{if } n =    1 \\
F_{n - 1} + F_{n - 2}  & \mbox{if } n \geq 2 \\
\end{array}
\right.
$ \\ \\

\noindent
A faster method to calculate Fibonacci is to use Matrix Exponentiation where it is defined as follows \\
\noindent
$
\begin{bmatrix}
F_{n + 1} \\
F_{n} \\
\end{bmatrix} =
\begin{bmatrix}
1 & 1 \\
1 & 0 \\
\end{bmatrix}
\begin{bmatrix}
F_{n} \\
F_{n - 1} \\
\end{bmatrix}
$ \\ \\

\noindent
It can be generalised to the following to obtain the nth term \\
$
\begin{bmatrix}
F_{n + 1} \\
F_{n} \\
\end{bmatrix} =
\begin{bmatrix}
1 & 1 \\
1 & 0 \\
\end{bmatrix}^{n - 2}
\begin{bmatrix}
F_{2} \\
F_{1} \\
\end{bmatrix}
$ \\
where $F_{2} = 1$ and $F_{1} = 1$ are the base cases

\subsection{Sieve of Eranthoses}

\subsection{Formulae}
\subsubsection{Polygon}
\begin{itemize}
\item area of triangle given 3 sides:  a = sqrt(p*(p-a)*(p-b)*(p-c)) where p = (a+b+c)/2
\item area of n sided regular polygon given length of 1 side,s: a = (s*s)*n/4tan(180/n)
\item area of n sided regular polygon given circumradius (radius of circle around polygon): \\ a = ((r*r)*n)*sin(360/n)/2
\item area of n sided regular polygon given inradius (radius of circle inside polygon) or the apothem (the perpendicular distance from center to a side): a = ((a*a)*n)*tan(180/n)
\item perimeter of a circle: 2*pi*r
\item area of irregular polygon: break into triangles by segmenting at vertices
\item Number of diagonals in a polygon is n sides: n*2(n-3)/2
\end{itemize}

\subsubsection{Radii}
\begin{itemize}
\item radius of a regular polygon (or the circumradius): r = s/(2*sin(180/n))
\item circumradius of a regular polygon given the apothem (apothem is inradius): r = a/(cos*(180/n))
\item inradius (apothem) given length of side, s: ir = s/(2*tan(180/n))
\item inradius given circumradius: ir = r*cos(180/n)
\end{itemize}

\subsubsection{Combinations and Permutations}


